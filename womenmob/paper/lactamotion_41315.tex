%%%%%%%%%%%%%%%%%%%%%%% file template.tex %%%%%%%%%%%%%%%%%%%%%%%%%
%
% This is a general template file for the LaTeX package SVJour3
% for Springer journals.          Springer Heidelberg 2010/09/16
%
% Copy it to a new file with a new name and use it as the basis
% for your article. Delete % signs as needed.
%
% This template includes a few options for different layouts and
% content for various journals. Please consult a previous issue of
% your journal as needed.
%
%%%%%%%%%%%%%%%%%%%%%%%%%%%%%%%%%%%%%%%%%%%%%%%%%%%%%%%%%%%%%%%%%%%
%
% First comes an example EPS file -- just ignore it and
% proceed on the \documentclass line
% your LaTeX will extract the file if required
\begin{filecontents*}{example.eps}
%!PS-Adobe-3.0 EPSF-3.0
%%BoundingBox: 19 19 221 221
%%CreationDate: Mon Sep 29 1997
%%Creator: programmed by hand (JK)
%%EndComments
gsave
newpath
  20 20 moveto
  20 220 lineto
  220 220 lineto
  220 20 lineto
closepath
2 setlinewidth
gsave
  .4 setgray fill
grestore
stroke
grestore
\end{filecontents*}
%
\RequirePackage{fix-cm}
%
%\documentclass{svjour3}                     % onecolumn (standard format)
%\documentclass[smallcondensed]{svjour3}     % onecolumn (ditto)
\documentclass[smallextended]{svjour3}       % onecolumn (second format)
%\documentclass[twocolumn]{svjour3}          % twocolumn
%
\smartqed  % flush right qed marks, e.g. at end of proof
%
\usepackage{graphicx}
%
% \usepackage{mathptmx}      % use Times fonts if available on your TeX system
%
% insert here the call for the packages your document requires
%\usepackage{latexsym}
% etc.
%
% please place your own definitions here and don't use \def but
% \newcommand{}{}
%
% Insert the name of "your journal" with
% \journalname{myjournal}
%
\begin{document}

\title{Lactamotion:%\thanks{Grants or other notes
%about the article that should go on the front page should be
%placed here. General acknowledgments should be placed at the end of the article.}
}
\subtitle{Postpartum mobility in northwestern Namibia}

%\titlerunning{Short form of title}        % if too long for running head

\author{Layne Vashro
}

%\authorrunning{Short form of author list} % if too long for running head

\institute{L. Vashro \at
              270 South 1400 East, Salt Lake City, UT 84112 \\
              Tel.: +001 (801) 581 6251\\
              \email{layne.vashro@anthro.utah.edu}           %  \\
}

\date{Received: date / Accepted: date}
% The correct dates will be entered by the editor


\maketitle

\begin{abstract}
Insert your abstract here. Include keywords, PACS and mathematical
subject classification numbers as needed.
\keywords{First keyword \and Second keyword \and More}
% \PACS{PACS code1 \and PACS code2 \and more}
% \subclass{MSC code1 \and MSC code2 \and more}
\end{abstract}

\section{Introduction}
\label{sec:1}
Researchers consistently find differences between men and women in spatial-cognitive and navigational tasks, as well as measures of traveling range.  These differences are well-documented in Western industrialized societies and have increasingly been replicated cross-culturally (cite cite cite).  Evolutionary psychologists have put forward several distinct theories that link the sex differences across these traits into a single cohesive story.  In most of these theories, past selection favored the males who were better at traveling long distances and into unknown environments and this required superior navigation ability and the spatial-cognitive traits that facilitate it.  The key point of disagreement among these arguments is simply the presumed payoff of that travel (mates (cite), hunting (cite), or warfare (cite)?).  However, one explanation for the sex differences in ranging, spatial cognition, and navigation ignores the payoffs to males and instead turns the focus on the fitness ramifications of women's long-distance mobility.  This ``fertility and parental care hypothesis'' put forward by \cite{sherry1997evolution} argues that the observed sex differences can be explained in terms of the potential costs to women traveling, particularly during key periods of reproduction.

	\subsection{Fertility and parental care}
	\label{sec:1.1}
The fertility and parental care hypothesis draws from the patterning of women's performance on spatial tasks in concert with hormonal changes related to their reproductive cycle.  In particular, the negative correlation between estrogen levels and spatial ability (cite).  
	
Risky strategies tend to pay better fitness dividends when variance in reproductive success among competitors is high (cite cite cite).  As is the case in many species, men's reproduction skews higher than women's (cite).  The constraints of women's extensive prepartum investment in offspring places a stringent ceiling on potential reproduction, and in doing so limit the prospective bounty paid to risky strategies relative to men.  In addition to this common mammalian pattern, humans are a particularly altricial species with an extended period of infant dependence that falls predominantly on the mother in most societies.  Men balancing the trade-offs of risky decisions need to account for the loss of potential future offspring, but at least in the subsistence societies that have been investigated, their deaths do not appear to threaten the outcomes of living children.  This is not the case for mothers \cite{hill1996ache, sear2008keeps}.  The application of this theoretical perspective has played an important role in our understanding a variety of topics from sex-differences in violence.... even in economics \cite{wilson1985competitiveness}  \cite{wilson1997life} \cite{jianakoplos1998women}.

While less dramatic than some of the above examples, travel away from home is risky behavior.  Risks including large predators, snakes, inclement weather, exposure, falling rocks, and many other dangers are real concerns when navigating wild natural environments.  In many cases, interpersonal violence may also pose a serious cause for concern, especially when traveling along or in small groups.  Finally, their is a non-zero risk of simply becoming lost and never making your way back to a familiar place in a world without cellular phones and GPS navigation systems (cite cite).  This problem is likely most severe in dense jungle or open ocean contexts, but exists on some level in many wild environments.  The nature of the risk has changed for many of us in today's world, but travel remains one of the riskier activities.  Travel related ``road injury'' is the seventh most common cause of death worldwide \cite{krug2000global}, and even in the United States traffic accidents are the second largest external cause of death \cite{sherry2010cdc}.  

In addition to the risks associated with travel, the fertility and parental care hypothesis also notes the energetic costs of traveling and how these trade-off against the need to divert as many calories as possible towards reproduction (cite).  With these concerns about the risks and energetic costs of travel in mind, the link between hormonal patterns associated with women's reproduction and the tools and desire to travel broadly presents an appealing evolutionary narrative.  However, a number of key predictions remain untested.  

	\subsection{Predictions}
	\label{sec:1.2}
	
1.  Men will demonstrate higher spatial ability, report lower spatial anxiety, and travel more broadly. 
	
2.  Women in their reproductive years will demonstrate lower spatial ability, report higher spatial anxiety, and less.

3.  Reproductive-aged women will demonstrate lower spatial ability, report higher spatial anxiety, and less when pregnant or lactating.

4.  Spatial cognition will be a better predictor of women's ability than men's. 


Predation is one major travel-related risk that constrains behavior throughout throughout the animal kingdom (cite).  Human's are a relatively large species, but at least in some wild environments large predators may pose a legitimate threat to adults and especially children traveling in tow (cite).  Intraspecies aggression is another possible danger that is particularly relevant to some of our closest Great Ape ancestors. 



Female Gorillas (Gorilla gorilla gorilla) avoid transfers when they have infants \cite{watts1989infanticide, stokes2003female}.  Female Chimpanzees (Chimpanzee chimpanzee) typically disperse from their natal home when they reach reproductive maturity then never make a secondary transfer.  However, the threat of attacks from out-group males is a danger to females with dependent offspring travel the periphery of their home region. \cite{mitani2002recent, watts1989infanticide} (cite others also)








Looking across the entire life-cycle, it is true that women's estrogen level rise as they enter reproductive maturity (is this tru?), and fall after menopause.  This lines up with the theory linking decreased mobility as a risk reduction strategy that is mediated by estrogen.  However, looking within women of reproductive age the pattern of estrogen cycles is more difficult to link with at least a simple version of risk reduction.  The problem is that women's estrogen level drop post-partum.  The team between  birth and weaning is likely \emph{the} time when the concerns highlighted by the fertility and parental care hypothesis should be most important. Instead, at least among mice, this period is associated with improved performance in maze tasks ans increased range (nest visits?... check this). 


	
The fertility and parental care hypothesis makes several assumptions about the relationship between demographics and life history and cognition and mobility.  In many instances these assumptions are consistent with the observed pattern in Western industrialized nations, but have not been substantiated in natural fertility populations, or societies where mobility demands are more consistent with those humans faced throughout most of their evolutionary history.

1) Women should be less mobile and perform worse in spatial cognition tasks than men, and this difference should be particularly pronounced during peak years of fertility.  2) Women's mobility 



\section{Methods}
\label{sec:2}
	\subsection{Population}
This study includes all women living in the \emph{Ovizorowe} mountain valley in northwestern Namibia.  This sample includes mostly members of the Twe ethnic group who live in several villages dispersed throughout the valley, as well as women from some Himba and Tjimba communities bookending the valley. 

While the Kunene region is remarkably ``wild'' and ``untouched'' by modern standards, it is likely much safer to travel than many ancestral environments.  Leopards and poisonous snakes are a legitimate concern, but most of the dangerous predators have been eliminated from the region due tot the threat they pose to cattle.  The threat of interpersonal and tribal violence is also much lower than in the not so distant past.  That said, participants did report concern about meeting strangers during travel and I encountered one small village in the mountains where a rapist had been targeting women when they visited their garden unaccompanied. 

		\subsubsection{Mental rotation}
		\label{sec:2.2.1}		
The mental rotation task was based on the Mental Rotation Test (MRT) developed by \cite{allport171shepard}. In the traditional task, objects created by configurations of cubes are presented to the viewer. These cubes are either rotated versions of the same cube or an entirely different cube. The viewer is then asked to decide if the cubes are the same or different. Both accuracy and response time are recorded. We designed an adapted computer-based version of the MRT using gaming software \cite{unity14} and conducted the experiment on a Toshiba 15.6" Touch-Screen laptop.  Due to the novel nature of rectilinear objects, for the Twe, we adapted the Mental Rotation Test by having the viewers make judgments about computer generated images of human bodies with one out stretched arm, which were rotated on a two-dimensional axis. Additionally, instead of showing the viewers two objects and asking them to make a same or different decision, we chose to show one target image at the top of the screen and have the viewer select one of two images that matched the target. This configuration was designed to avoid linguistic issues around having a button or icon that indicated same or different. Instead, the viewer simply used the touched screen interface to touch the body that matched the target object. 12 trials were presented to the viewers showing the bodies from the front and back, rotated twice at 0, 60, 120, 180, 240, and 300 degrees, making a total of 24 trials. This task was readily understood by most participants. However, it was unclear if some participants understood the task, as a few adopted a strategy where they only selected the left or right responses or alternated between the two. For this reason, only participants that scored above chance were used in the analyses. 

		\subsubsection{Real-world pointing}
		\label{sec:2.2.2}
We used Real-world distant pointing accuracy as a measure of navigation abilities. Previous work has suggested that the ability to point to a location is a skill that is uniquely develop by mobility \cite{bell2004relationship}. In our task we selected XX locations that were designed to form a normal distribution of places that the Twe had likely visited with varying distances. We used a Brunton Pocket Transit International Compass with 0-360 Degree Scale mounted on a tripod to record the pointing data. Viewers were first asked if they had visited one of the locations. If they had, they were then asked to use the sight on the compass to point to the location. Degree measurements were then taken from the compass. Accuracy was measured by taking the GPS coordinates that the compass was located at and the coordinates of the location pointed to, then calculating the degree of error from were the viewer pointed and the actual location. This resulted in errors ranging from 0 degrees (perfectly accurate) to 180 degrees (completely in the wrong direction). Measurements were taken in locations that were free of objects that visually occluded participants’ views (e.g. dense foliage and mountains).

	\subsection{Mobility interviews}
	\label{sec:2.1}
Participants were asked to name each place they traveled to and spent the night in the past year.  In addition, they were asked who they traveled with, who they stayed with, and why they made the trip.  These data were used to create two highly-correlated measures of annual range size.  1) The number of unique places visited in the past year, and 2) the number of kilometers needed to visit each place visited in the past year.  In addition, the analysis is able to discriminate between pair, group, and unaccompanied travel, and identify patterning in the function of travel.

\section{Results}
\label{sec:3}
Your text comes here. Separate text sections with	

	\subsection{Comparisons across demography and life history}
	\label{sec:3.1}
		\subsubsection{Spatial Cognition}		
		\label{sec:3.1.1}
		
According to predictions drawn from the fertility and parental care hypothesis, we should find that males perform better than women and women with unweaned infants perform worse than other reproductive-aged women.  We should also expect postmenopausal women to perform better than women still in the fertile stage of their career.

Men responded more accurately but slightly slower in the mental rotation task (see Table \ref{tab:cog}).  Postmenopausal women responded slower than reproductive-aged women and were slightly less accurate.  Actual expectations regarding the performance of postmenopausal women may be more complex than presented here, since there are likely confounding forces depressing performance of the older postmenopausal women.  However, comparing men split into analogous age groups the differences look similar but with a slightly weaker decline (Accuracy decrease from 89.6\% to 86.7\% and reaction time increase from 5.7 to 7.6).  It does not look like women's spatial ability improves after menopause even accounting for general age-based decline shared with men.  Breastfeeding women responded slightly quicker and more accurately to the mental rotation task than other women of reproductive age, but the differences are small enough to easily be explained by random chance.

\begin{table}[h!]
\caption{Spatial Cognition}
\label{tab:cog}  
\begin{tabular}{lllllll}
\hline\noalign{\smallskip}
Comparison & \phantom{0}n$|$n & $\mu1$ & $\mu2$ & d & \multicolumn{2}{c}{95\% CI} \\
\noalign{\smallskip}\hline\noalign{\smallskip}
Men $|$ Women & 55$|$43 & 89.3\% & 82.7\% & \phantom{-}0.457* & \phantom{-}0.043 & \phantom{-}0.871 \\
& & 5.9 & 5.6 & \phantom{-}0.149 & -0.260 & \phantom{-}0.558 \\
Postmenopausal $|$ Not & \phantom{0}5$|$38 & 77.1\% & 83.4\% & -0.384 & -1.372 & \phantom{-}0.604 \\
& & 7.5 & 5.4 & \phantom{-}1.279* & \phantom{-}0.253 & \phantom{-}2.306 \\
Lactating $|$ Not & 21$|$14 & 83.7\% & 81.0\% & \phantom{-}0.163 & -0.561 & \phantom{-}0.888 \\
& & 5.4 & 5.8 & -0.276 & -1.002 & \phantom{-}0.450 \\
\noalign{\smallskip}\hline
\end{tabular}\par
\bigskip
d is Cohen's d measure of effect size \cite{cohen1988statistical}. First row under each group gives accuracy, and second row gives reaction times.  Comparisons where the 95\% confidence interval does not include 0 are denoted with a ``*''.
\end{table}		  

The patterning of participants dropped due to insufficient understanding complicates these findings.  The participants who were removed from the sample either due to failure in the practice or ultimately scoring below chance on the task were more likely to be women, specifically post-menopausal women.  Only 18.8\% of men who were shown the practice failed to demonstrate understanding compared to 28.3\% of the women.  Within the set of 64 women, 61.\% of the 16 post-menopausal participants failed to demonstrate understanding compared to 21.4\% of the younger women.  This limited understanding may be a function of some cognitive feature(s) unrelated to the spatial cognitive abilities in question, however, to whatever extent the traits of interest help explain their omission these results understate the difference between reproductive-aged and post-menopausal women's spatial cognition.


		\subsubsection{Navigation}
		\label{sec:3.1.2}

According to predictions drawn from the fertility and parental care hypothesis, we should find that males perform better than women and women with unweaned infants perform worse than other reproductive-aged women.  We should also expect postmenopausal women to perform better than women still in the fertile stage of their career.

Men outperformed women in the real-world pointing measure of navigational skill (see Table \ref{tab:nav}).  There do not appear to be any meaningful differences between postmenopausal and younger women in pointing accuracy.  Post-partum women performed considerably better than other reproductive-aged women, however, the 95\% confidence interval around the Cohen's D measure includes 0.  A larger sample may be needed to assess the relationship between breastfeeding and navigation ability.

\begin{table}[h!]
\caption{Navigation}
\label{tab:nav}  
\begin{tabular}{lllllll}
\hline\noalign{\smallskip}
Comparison & \phantom{0}n$|$n & $\mu1$ & $\mu2$ & d & \multicolumn{2}{c}{95\% CI} \\
\noalign{\smallskip}\hline\noalign{\smallskip}
Men $|$ Women & 61$|$57 & 15.18$^{\circ}$ & 19.22$^{\circ}$ & -0.481* & -0.855 & -0.108 \\
Postmenopausal $|$ Not & 14$|$43 & 20.54$^{\circ}$ & 18.79$^{\circ}$  & \phantom{-}0.188 & -0.441 & \phantom{-}0.816\\
Lactating $|$ Not & 24$|$17 & 16.73$^{\circ}$ & 20.97$^{\circ}$  & -0.422 & -1.087 & \phantom{-}0.243 \\
\noalign{\smallskip}\hline
\end{tabular}\par
\bigskip
Comparisons where the 95\% confidence interval does not include 0 are denoted with a ``*''.
\end{table}		  

		\subsubsection{Anxiety}
		\label{sec:3.1.2}
According to predictions drawn from the fertility and parental care hypothesis, we should find that males self-report lower spatial anxiety than women and women with unweaned infants report higher anxiety than other reproductive-aged women.  We should also expect postmenopausal women to report lower anxiety than women still in the fertile stage of their career.

Consistent with expectations, men reported lower spatial anxiety than women (see Table \ref{tab:anx}).  The other comparisons also trend in the expected directions with postmenopausal women expressing lower spatial anxiety than younger women and postpartum women reporting more anxiety than their counterparts.  However, the samples for each of these latter comparison are very small, which allows the 95\% confidence interval around the Cohen's D to encompass 0 despite very strong effect sizes.

\begin{table}[h!]
\caption{Spatial Anxiety}
\label{tab:anx}  
\begin{tabular}{lllllll}
\hline\noalign{\smallskip}
Comparison & \phantom{0}n$|$n & $\mu1$ & $\mu2$ & d & \multicolumn{2}{c}{95\% CI} \\
\noalign{\smallskip}\hline\noalign{\smallskip}
Men $|$ Women & 27$|$27 & 2.29 & 2.64 & -0.743* & -1.325 & -0.161 \\
Postmenopausal $|$ Not & \phantom{0}8$|$19 & 2.45 & 2.72  & -0.773 & -1.705 & \phantom{-}0.159\\
Lactating $|$ Not & 12$|$5 & 2.83 & 2.60 & \phantom{-}0.971 & -0.302 & \phantom{-}2.244 \\
\noalign{\smallskip}\hline
\end{tabular}\par
\bigskip
Comparisons where the 95\% confidence interval does not include 0 are denoted with a ``*''.
\end{table}		 

		\subsubsection{Mobility}
		\label{sec:3.1.2}
Consistent with expectation, men are much more mobile than women.  Men had visited more than twice as many unique locations in the past year and those trips were almost twice as likely to be made without a companion (see Table \ref{tab:mob}).  Men also traveled slightly more than twice as far as women on a day-to-day basis around the home region.  

\begin{table}[h!]
\caption{Mobility}
\label{tab:mob}  
\begin{tabular}{lllllll}
\hline\noalign{\smallskip}
Comparison & \phantom{0}n$|$n & $\mu1$ & $\mu2$ & d & \multicolumn{2}{c}{95\% CI} \\
\noalign{\smallskip}\hline\noalign{\smallskip}
Men $|$ Women & 42$|$45 & 4.29 & 2.02 & \phantom{-}0.725* & \phantom{-}0.280 & \phantom{-}1.171 \\
& 40$|$40 & 46.4\% & 24.2\% & \phantom{-}0.586* & \phantom{-}0.125 & \phantom{-}1.046 \\
& 20$|$18 & 8.75km & 4.38km & \phantom{-}1.000* & \phantom{-}0.280 & \phantom{-}1.720 \\
Postmenopausal Not & 10$|$35 & 1.60 & 2.14 & -0.341 & -1.085 & \phantom{-}0.402 \\
& 10$|$30 & 35.0\% & 20.6\% & \phantom{-}0.390 & -0.374 & \phantom{-}1.154 \\
& \phantom{0}3$|$15 & 7.05km & 3.85km & \phantom{-}1.361 & -0.161 & \phantom{-}2.882 \\
Lactating $|$ Not & 20$|$12 & 2.80 & 1.33 & \phantom{-}0.899* & \phantom{-}0.091 & \phantom{-}1.707 \\
& 19$|$9 & 22.0\% & 22.2\%  & -0.008 & -0.872 & \phantom{-}0.855 \\
& 19$|$5 & 3.78km & 3.71km & \phantom{-}0.035 & -1.395 & \phantom{-}1.464 \\
\noalign{\smallskip}\hline
\end{tabular}\par
\bigskip
First row under each group gives accuracy, and second row gives reaction times.  Comparisons where the 95\% confidence interval does not include 0 are denoted with a ``*''.
\end{table}		

The fertility and parental care hypothesis predicts that postmenopausal women should be more mobile than women in the midst of their reproductive careers.  Postmenopausal Twe women actually reported visiting slightly fewer places in the past year (a decline that is nearly identical to that of similar-aged men), however, they did make a higher percentage of their trips unaccompanied which is consistent with the expectation of diminished risk-aversion.  This difference small enough that the 95\% confidence interval around the Cohen's D easily includes 0.  Among the three post-menopausal women to participate in the daily task, one recorded the highest average travel of all eighteen women included in the study (11.22 km), while the other two older women averaged a kilometer more daily travel than the average of the reproductive-aged women (4.97 km compared to 3.85 km).  A larger sample is clearly needed, but these initial findings are intriguing and consistent with expectation drawn from the fertility and parental care hypothesis.

We expected postpartum women to curtail their mobility due to the risks and caloric costs of travel.  However, Twe women with unweaned children visited more than twice as many locations as other reproductive-aged women in the past year (see Figure \ref{fig:bfeed_tot}).  These visits were no more or less likely to be solo ventures.  There also does not appear to be any difference in daily travel. 		

% For one-column wide figures use
\begin{figure}[!htb]
  \includegraphics[width=0.75\textwidth]{bfeed_tot}
\caption{Please write your figure caption here}
\label{fig:1}       % Give a unique label
\end{figure}

	\subsection{Prediction 4: ... }
	\label{sec:3.1}
The fertility and parental care hypothesis predicts a positive relationship between spatial-cognitive ability and mobility. This expectation is shared with the other prominent theories linking spatial cognition to travel-based fitness effects, however, the others focus on this relationship in men rather than women.  Thus, looking at which sexes travel more in response to variance in spatial ability may help discriminate between possible explanations.	

\begin{table}[!tb]
\caption {Annual mobility and Spatial Cognition}
\label{tab:wm_spacemob}
  \centering
  \begin{tabular}{| l  ll  ll  ll  l |} 
    \hline   
  & \multicolumn{6}{c}{Independent Variables}& \\    
  & \multicolumn{2}{c}{MR} & \multicolumn{2}{c}{Male(1$|$0)} & \multicolumn{2}{c}{Male(1$|$0):MR} & $R^2$ \\
  & $Std. \beta$ & $Std. Err$ & $Std. \beta$ & $Std. Err$ & $Std. \beta$ & $Std. Err$  & \\
  Model 1 & 0.207\phantom{**} & 0.134\phantom{**} & & & & & 0.036\phantom{**} \\
  Model 2 & 0.262. & 0.137\phantom{**} & 0.331** & 0.114\phantom{**} & .300* & 0.131\phantom{**} & 0.222\\
  \hline 
  \end{tabular}  
{}
\end{table}

Mental rotation performance alone is only weakly predictive of travel in the past year and is not a statistically significant improvement over a null model ($M_{null} | M_1$, $\chi^2(1, 98) = 2.348, p = 0.121$).  However, including sex as an interaction effect dramatically improves model performance ($M_1 | M_2$, $\chi^2(2, 98) = 12.091, p = 0.0006$).  Interestingly, the direction of the effect is in the opposite direction of expectations drawn from the fertility and parental care hypothesis.  Men, but not women, with higher spatial ability appear to travel more broadly (see Figure \ref{fig:wm_acctot} and Table \ref{tab:wm_spacemob}).  This is consistent with findings in a previous study using a different measure of mental rotation (cite me).

% For one-column wide figures use
\begin{figure}[!htb]
  \includegraphics[width=0.75\textwidth]{mw_acctot}
\caption{Please write your figure caption here}
\label{fig:1}       % Give a unique label
\end{figure}

\section{Discussion}
\label{sec:4}

The observed sex differences across spatial cognition, navigation, spatial anxiety, annual mobility and daily mobility are all consistent with the fertility and parental care hypothesis.  Men outperformed women in the spatial-cognitive and navigational tasks, reported lower spatial anxiety, and traveled further at both scales.  However, all of these predictions apply equally well to the other prominent theories linking these traits in an evolutionary framework.  

The only area of this study that consistently fits expectations uniquely drawn from the fertility and parental care hypothesis is the spatial anxiety measure.  Postmenopausal women reported much lower anxiety than reproductive-aged women, and among the latter group, women currently dealing with unweaned infants reported much higher anxiety.  Unfortunately, both of these tests lacked the power necessary to confidently reject random variance as an explanation for the patterning.  The mobility data also holds intriguing trends in the difference between postmenopausal and reproductive-aged women, with the older women moving much more on a daily basis and making a higher percentage of their annual visits abroad without accompaniment.  These trends are consistent with the fertility and parental care hypothesis but again lack statistical power.

Interestingly, one of the strongest findings of the study actually runs in the opposite direction of the fertility and parental care hypothesis.  Women with nursing infants traveled to more than twice as many unique locations as other reproductive-aged (and not pregnant) women in the past year.  The period between childbirth and the successful weaning of an infant is arguably the most vulnerable time in a woman's life in terms of travel risks.  Our measure of spatial anxiety shows women at this stage may feel appreciate things concerns. (does anxiety moderate??  I think it does)  However, these women actually expand on their annual travels in spite of this anxiety.  

The additional 

Scelza (cite) found that nearby Himba women travel more  

One possible explanation for women's increased travel during the postpartum period is that they are trips   

along these same lines.  Women with unweaned infants also performed better than other same-aged women in the real-world pointing measure of navigational ability.

Comparing reproductive-aged women living in the \emph{Ovizorowe} Valley with and without nursing depends highlights interesting differences in both cognition and mobility.  Breastfeeding women performed better across our measures of spatial cognition (excepting the perspective-taking task), which is consistent with expectations based on the decline in estrogen associated with the post-partum period.  In addition, breastfeeding women were more mobile, traveling to more unique locations and covering more ground in doing so than their peers who were without an unweaned child over that period of time.  This increase in mobility may also be consistent with the down-tick in estrogen and improved spatial cognition from the perspective of several theories linking spatial cognitive ability to distant ranging.  

These findings simultaneously support the patterns of cognition and behavior anticipated by the fertility and parental care hypothesis while complicating the interpretation with the fact that women are most mobile exactly when it seems least likely from the perspective of minimizing risk to offspring.

Why would it be beneficial for women to range further when they have unweaned children?  One study among conducted among a nearby  Himba population (Scelza) showed women traveling the most to visit their mothers during periods of peak childcare need.  This seems an appealing answer to this situation as well, however, most of these women were actually moving \emph{away} from their mothers (is this true??) as a much greater fraction of this population lives matrilocally.  In addition, none of the women explicitly cited visiting their mother.  Several other alternatives... 1) ``Facebook'' effect...  ``Look at the baby, look at the baby''... making connections to relatives who may be called on in future times of need.  2) Rape deterred... then what about pregnancy??  Don't like this.

Interested in future work looking at the volume and function of women's postpartum mobility in other populations, including the US.


Text with citations \cite{RefB} and \cite{RefJ}.



%\begin{acknowledgements}
%If you'd like to thank anyone, place your comments here
%and remove the percent signs.
%\end{acknowledgements}

% BibTeX users please use one of
%\bibliographystyle{spbasic}      % basic style, author-year citations
%\bibliographystyle{spmpsci}      % mathematics and physical sciences
%\bibliographystyle{spphys}       % APS-like style for physics
%\bibliography{}   % name your BibTeX data base

% Non-BibTeX users please use
\begin{thebibliography}{}
%
% and use \bibitem to create references. Consult the Instructions
% for authors for reference list style.
%
\bibitem{RefJ}
% Format for Journal Reference
Author, Article title, Journal, Volume, page numbers (year)
% Format for books
\bibitem{RefB}
Author, Book title, page numbers. Publisher, place (year)
% etc
\end{thebibliography}

\end{document}
% end of file template.tex

