%----------------------------------------------------------------------------------------
%	DOCUMENT CONFIGURATIONS
%----------------------------------------------------------------------------------------

\documentclass{letter}

\usepackage{graphicx}

% Adjust margins for aesthetics
\addtolength{\voffset}{-0.5in}
\addtolength{\hoffset}{-0.3in}
\addtolength{\textheight}{2cm}

%\longindentation=0pt % Un-commenting this line will push the closing "Sincerely," to the left of the page

%----------------------------------------------------------------------------------------
%	YOUR NAME & ADDRESS SECTION
%----------------------------------------------------------------------------------------

\address{270 S. 1400 E. Rm 102 \\ Salt Lake City, Utah 84112  \\ (612) 619-4820} % Your address and phone number

%----------------------------------------------------------------------------------------

\begin{document}

%----------------------------------------------------------------------------------------
%	ADDRESSEE SECTION
%----------------------------------------------------------------------------------------

\begin{letter}{} % Name/title of the addressee

%----------------------------------------------------------------------------------------
%	LETTER CONTENT SECTION
%----------------------------------------------------------------------------------------

\opening{\textbf{Dear Dr. Lancaster,}}
 
Please find enclosed a manuscript entitled: “Childcare Dynamics and Women's Residential Autonomy”

The paper demonstrates how a woman’s best source of childcare assistance is likely to shift from her own mother to her early-born daughters as she progresses through her reproductive career.  This shift is important to understanding posmarital residence, because the former source is tied to the natal camp while the latter offers a mobile alternative and frees mothers to respond to other residence incentives.  I use a computer simulation to highlight this pattern and argue its importance for explaining the delayed female dispersal which is common among hunter-gatherer societies.  This paper should appeal to readers interested in postmarital residence, bride-service, or the broader animal literature on dispersal and philopatry, as well as readers interested in cooperative breeding, particularly the role of ``helpers at the nest.'' 

This article has not been submitted to any other journals and represents no conflicts of interest for the author.

Thank you for your consideration.  Please address all correspondence concerning this manuscript to me by email at (Layne.Vashro@anthro.utah.edu)


\vspace{2\parskip} % Extra whitespace for aesthetics
\closing{Sincerely,\\
\fromsig{\includegraphics[scale=.175]{signature}} \\
\fromname{Layne J. Vashro}
}

%\ps{P.S. You can find additional information attached to this letter.} % Postscript text, comment this line to remove it

%\encl{Copyright permission form} % Enclosures with the letter, comment this line to remove it

%----------------------------------------------------------------------------------------

\end{letter}
 
\end{document}