grestore
stroke
grestore
\end{filecontents*}
%
\RequirePackage{fix-cm}
%
%\documentclass{svjour3}                     % onecolumn (standard format)
%\documentclass[smallcondensed]{svjour3}     % onecolumn (ditto)
\documentclass[smallextended]{svjour3}       % onecolumn (second format)
%\documentclass[twocolumn]{svjour3}          % twocolumn
%
\smartqed  % flush right qed marks, e.g. at end of proof
%
\usepackage{graphicx}
\usepackage{apacite}
\usepackage{natbib}
%
% \usepackage{mathptmx}      % use Times fonts if available on your TeX system
%
% insert here the call for the packages your document requires
%\usepackage{latexsym}
% etc.
%
% please place your own definitions here and don't use \def but
% \newcommand{}{}
%
% Insert the name of "your journal" with
% \journalname{myjournal}
%
\begin{document}

\title{Childcare Dynamics and Women's Residential Autonomy%\thanks{Grants or other notes
%about the article that should go on the front page should be
%placed here. General acknowledgments should be placed at the end of the article.}
}
\subtitle{Do you have a subtitle?\\ If so, write it here}

%\titlerunning{Short form of title}        % if too long for running head

\author{Layne Vashro
}

%\authorrunning{Short form of author list} % if too long for running head

\institute{L. Vashro \at
              270 S. 1400 E. Rm 102 Department of Anthropology \\
              Tel.: 801-581-6251\\
              \email{layne.vashro@anthro.utah.edu}           %  \\
}

\date{Received: date / Accepted: date}
% The correct dates will be entered by the editor


\maketitle

\begin{abstract}
Insert your abstract here. Include keywords, PACS and mathematical
subject classification numbers as needed.
\keywords{First keyword \and Second keyword \and More}
% \PACS{PACS code1 \and PACS code2 \and more}
% \subclass{MSC code1 \and MSC code2 \and more}
\end{abstract}

\section{Introduction}
\label{intro}
Females among humans' closest primate relatives disperse from their natal group after reaching reproductive maturity \cite{nishida1987chimpanzees, gerloff1999intracommunity, stokes2003female}.  This is also the case in many human societies \cite{murdock1967ethnographic}, but is far from a universal pattern and only describes a minority of documented foragers \cite{alvarez2004residence, marlowe2004marital}.  Instead, the modal pattern of post-marital residence among human foragers is “multilocal,” where both sexes may or may not disperse from their natal camp \cite{hill2011co}.  These flexible systems are complex and situation specific, but often women remain in their family's household initially then later move away, possibly into the familial household of their husbands \cite{marlowe2004marital}.  This general pattern is supported by informants in a wide variety of forager societies \cite{lee1974male, hewlett1993intimate, marlowe2004marital, hill1996ache}, but has only been assessed with quantitative data in a minority of cases (see \cite{hill2011co_sup} for the Ach\'{e} and Ju/’hoansi and \cite{jones2005hadza} for the Hadza).  Marlowe finds that twenty-five percent of the forager groups in the Standard Cross Cultural Sample fit this pattern \cite{marlowe2004marital, murdock1969standard}, however the real prevalence may be even higher since the ethnographic data rarely gives the level of detail needed to identify age-based trends in “multilocal” residence.  In this paper, I use a computer simulation to demonstrate how this pattern could follow from the dynamics of childcare assistance within the natal camp over time. 

Due to short inter-birth intervals and long periods of childhood dependence, human mothers often find themselves simultaneously caring for multiple young children \cite{key2000evolution}.  In order to meet this challenge, dependence on non-maternal sources of care is an important feature of human life-history \cite{hawkes1998grandmothering, kaplan2000theory}.  Children require food, physical contact, transportation, washing, and constant attention to ensure their safety.  These needs are particularly acute in the first several years of life when mortality risk is at its peak \cite{hill1996ache}.  Fathers often add support in the form or food provisioning and in some cases holding infants \cite{marlowe2003critical, crittenden2008allomaternal}, but are much less involved in other aspects of childcare.  This may be due to the sexual division of labor which separates men from their children throughout much of the day \cite{hewlett1992father}.  This lack of direct childcare could explain why the loss of a father has little or no effect on child survival across many different societies \cite{sear2008keeps}.  Especially when food is widely shared throughout camp, the particular nature of paternal provisioning may be easily substituted by alternative sources when necessary \cite{sugiyama2005juvenile, kaplan1984food, hawkes2001hadza}.  Rather than fathers, it is often mothers' female relatives who have the strongest impact on children's well-being \cite{sear2008keeps}.  A woman's mother and sisters help by directly caring for her children and cooperatively managing the balance between watching over children and accomplishing daily tasks \cite{valeggia2009changing, henry2005child, crittenden2008allomaternal, hawkes1997hadza}.  Since these sources of care are available to women living in their familial camp, but not to those who move to live with their husbands' family, childcare assistance may be one important incentive for women to stay home.  

However, as early-born children age, those children also begin to play an important role in childcare assistance.  Even children as young as three years old begin offering rudimentary childcare and by six may be regularly responsible for tasks like carrying and watching over their younger siblings \cite{zeller1987role, kramer2004reconsidering, hames1988allocation, denham1974infant, dunbar2002helping}.  The role of older siblings in childcare varies cross culturally but in many places sisters, and in at least one case brothers, are responsible for more direct childcare than any source outside the mother \cite{kramer2005children}.  This sibling care is linked to positive outcomes like decreased inter-birth intervals, extended fertile period, increased rates of child survival, and increased leisure time for mothers \cite{turke1988helpers, Bereczei2002helping, crognier2001helpers, bove2002girl}.  Sibling care is interesting with respect to residence because unlike a woman's mother and sisters, her children are tied to \textit{her} rather than the familial household.  This makes older children a source of childcare assistance that can be utilized even after moving away from home.  This paper presents a computer simulation abstracted from the demographics and childcare decisions of a female kin-group across a twenty-year generation.  The findings demonstrate how the burden of childcare can shift away from aunts and grandmothers and onto the mother and older siblings over time, opening up the opportunity for women to move away from their familial household as they age.

\section{Section title}
\label{sec:1}
Text with citations \cite{RefB} and \cite{RefJ}.
\subsection{Subsection title}
\label{sec:2}
as required. Don't forget to give each section
and subsection a unique label (see Sect.~\ref{sec:1}).
\paragraph{Paragraph headings} Use paragraph headings as needed.
\begin{equation}
a^2+b^2=c^2
\end{equation}

% For one-column wide figures use
\begin{figure}
% Use the relevant command to insert your figure file.
% For example, with the graphicx package use
  \includegraphics{example.eps}
% figure caption is below the figure
\caption{Please write your figure caption here}
\label{fig:1}       % Give a unique label
\end{figure}
%
% For two-column wide figures use
\begin{figure*}
% Use the relevant command to insert your figure file.
% For example, with the graphicx package use
  \includegraphics[width=0.75\textwidth]{example.eps}
% figure caption is below the figure
\caption{Please write your figure caption here}
\label{fig:2}       % Give a unique label
\end{figure*}
%
% For tables use
\begin{table}
% table caption is above the table
\caption{Please write your table caption here}
\label{tab:1}       % Give a unique label
% For LaTeX tables use
\begin{tabular}{lll}
\hline\noalign{\smallskip}
first & second & third  \\
\noalign{\smallskip}\hline\noalign{\smallskip}
number & number & number \\
number & number & number \\
\noalign{\smallskip}\hline
\end{tabular}
\end{table}


%\begin{acknowledgements}
%If you'd like to thank anyone, place your comments here
%and remove the percent signs.
%\end{acknowledgements}

% BibTeX users please use one of
%\bibliographystyle{spbasic}      % basic style, author-year citations
%\bibliographystyle{spmpsci}      % mathematics and physical sciences
%\bibliographystyle{spphys}       % APS-like style for physics
%\bibliography{}   % name your BibTeX data base

% Non-BibTeX users please use
\begin{thebibliography}{}
%
% and use \bibitem to create references. Consult the Instructions
% for authors for reference list style.
%
\bibitem{RefJ}
% Format for Journal Reference
Author, Article title, Journal, Volume, page numbers (year)
% Format for books
\bibitem{RefB}
Author, Book title, page numbers. Publisher, place (year)
% etc
\end{thebibliography}

\end{document}
% end of file template.tex

