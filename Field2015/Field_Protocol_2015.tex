\documentclass{article} 

\usepackage{booktabs} % Required for the top and bottom rules in the table
\usepackage{float} % Required for specifying the exact location of a figure or table
\usepackage{graphicx} % Required for including images
\usepackage{lipsum} % Used for inserting dummy 'Lorem ipsum' text into the template

\newcommand{\HRule}{\rule{\linewidth}{0.5mm}} % Command to make the lines in the title page
\setlength\parindent{0pt} % Removes all indentation from paragraphs
\setlength{\parskip}{1em}

%---------------------------------------------------------------------------------------

\begin{document}

%----------------------------------------------------------------------------------------
%	TITLE PAGE
%----------------------------------------------------------------------------------------

\title{
\begin{center}
\HRule \\[0.4cm]
{\Huge \bfseries Fieldwork Protocol \\[0.5cm] \Large Summer 2015}\\[0.4cm] % Degree
\HRule \\[1.5cm]
\end{center}
}
\author{\Huge SCAN Project\\ \\ \LARGE http://www.scanproject.org/ \\[2cm]} % Your name and email address
\date{6 June 2015} % Beginning date
\maketitle

%----------------------------------------------------------------------------------------

\newpage

\tableofcontents

\newpage

%----------------------------------------------------------------------------------------

\section{Cognition}

\subsection{Mental Rotation}

\begin{itemize}

\item Open ``Practice'' application.

\item Select screen resolution of ``1366 X 768'' and ``Good'' graphics quality.

\item ``Your task is to decide if the body on the top matches either this body [\emph{point to left}] or this body [\emph{point to right}]. They match if the same arm on the top and bottom body is pointed out in the same direction. For example [\emph{point to screen}], this body on top matches this body [\emph{point to left or right depending on answer}] but not this one [\emph{point to other body}].  [\emph{Press the correct answer and get the tone feedback}][\emph{Tap the screen to go to the next trial}]. The body may be rotated, so you need to think about the body rotating back to upright to decide about the match. For example, to decide which body this matches [\emph{point to top}], you would think about it rotating like this [\emph{press the correct response so that they see it rotating}] and then you decide that it matches this one [\emph{point to correct response}][\emph{Tap the screen to go to the next trial}].''

``Now, you try it [\emph{this should be the 3rd practice trial}]. Here is a rotated body. Decide whether it matches this one [\emph{point to left}] or this one [\emph{point to right}]. When you have decided, press that body. [\emph{When they press, they will see the rotating target and will have feedback on whether it is correct or not. Explain that the correct tone is high pitch and the incorrect tone is low pitch }]. Let’s try it again [\emph{go on to 4th trial}]. Decide whether it matches this one [\emph{point to left}] or this one [\emph{point to right}][\emph{Participant responds}]. [\emph{This would complete the first 4 trials, and then move on to the next 4, for testing for criterion}].''

``Here are a few more to practice on. As soon as you decide which one the top body matches, touch the answer. Remember that it is important to touch your answer as quickly as you can, as soon as you have made the decision. [\emph{Have them complete this on their own}], with the feedback from the rotation. If they get 3 out of 4 correct, move on to the actual trials. If not, restart the practice file until they perform to criterion (3 out of 4).''

\item Open ``Trial'' application.

\item Select screen resolution of ``1366 X 768'' and ``Good'' graphics quality.

\item Enter subject ID, Date, Age, Sex, and ``1'' for the number of blocks.

\item ``Now you will do this task a few more times. This time you will not see the top body rotate after you make a decision. Still try to decide which one matches to the top body. Press the cross when you are reading to go on to the next one. Remember to try to respond as quickly as you can. Press the body that is your answer as soon as you decide.''

\item When the participate indicates he/she is ready.  Touch the screen to begin the trial.

\end{itemize}

%-----------------------------------------

\subsection{Corsi Blocks} % Multiple experiments can be included in a single day, this allows you to segment what was done each day into separate categories

[Needs to be adapted for the forthcoming version.]

\begin{itemize}

\item ``You will see some blue squares on the computer screen that will look like this [\emph{Point to the squares on the screen}].  The squares will turn yellow one at a time in an order. I would like you to remember the order that they turned yellow. When they are finished turning yellow you will need to touch each square in the same order they turned yellow. Here is an example [\emph{Press down arrow}]. [\emph{Note: This is a pdf file made to look like the experiment, so you need to control it by hand by pressing the down arrow to move on to another page.}]''

``Do you see how this one is now yellow? Different squares will turn yellow each time.  Do your best to remember which squares turned yellow and the order that they turned yellow.[\emph{Press down arrow and squares will be blue again.}] Please practice touching the square that turned yellow. [\emph{Tell them if they are right or wrong. The squares will not turn yellow as they press them in this practice, although they will in the real experiment.}]''

``Now more than 1 square will turn yellow. Try to remember which squares turn yellow and the order that they turn in. [\emph{Press down 3 times slowly to show the different squares turning yellow.}] Please touch the squares that turned yellow in the order that they turned in. [\emph{Tell them if they were right or wrong. If they seem to understand move on to the experiment}]''

\end{itemize}


%\begin{figure}[H] % Example of including images
%\begin{center}
%\includegraphics[width=0.5\linewidth]{example_figure}
%\end{center}
%\caption{Example figure.}
%\label{fig:example_figure}
%\end{figure}

%-----------------------------------------

\subsection{Rod and Frame}

To be added.

%----------------------------------------------------------------------------------------

\section{Mobility}

\subsection{Daily mobility}

\begin{itemize}

\item Label trackers.

\item Build a tracker-tracker worksheet

	\begin{itemize}
	\item Table \ref{tab:tracktrack} is an example worksheet for keeping track of tracker status.
	\item Each row represents a tracker, and uses the device's label as an ID.
	\item The second column should be updated to indicate the tracker's status (i.e. is it out with a participant? ready to leave?, currently charging? broken? lost?).
	\item The next three columns repeat for each additional participant to handle each device.  The first participant is listed as PID1, then next as PID2 and so on...
	\item For each participant, we indicate their name of ID, the exact time and date they receive the device, and the exact time and date they return the device.
	\item It may also be advisable to include an additional column to make notes for each participant when necessary.
	\end{itemize}

\begin{table}[H]
\begin{tabular}{l l l l l l l l}
\toprule
\textbf{ID} & \textbf{Status} & \textbf{PID1}  & \textbf{OUT1}  & \textbf{IN1} & \textbf{PIDn}  & \textbf{OUTn}  & \textbf{INn}\\
\toprule
1 & ready & v223 & $1015.25jun15$ & $1535.28jun15$ & & & \\
2 & chrg & t173 & $0720.20jun15$ & $2000.27jun15$ & x095 & $1937.28jun15$ & \\
3 & out & z515 & $0800.01jul15$ & & & \\
\bottomrule
\end{tabular}
\label{tab:tracktrack}
\end{table}

\item Prepare tracker, show it to participant, then ask the participant to affix the device somewhere out of the way.  Inform the participant that you would like them to carry it for N days.

\item Recover tracker. Export entire track as a single .gpx and .csv file in the format POPULATION\_EXPERIMENTER\_DATEBGN\_DATEEND\_TRACKERID  

ex: TWE\_LJV\_25jun15\_28jun15\_v223

\end{itemize}

%----------------------------------------------------------------------------------------

\subsection{Annual mobility}

\begin{itemize}

\item ``Which places away from home did you visit and stay the night at during the past year? Did you visit that place more than once? '' [\emph{record all places}]

\item Once you have a list of locations move through the list starting with ``Tell me about the most recent time you visited X,'' then moving on to the enumerated questions below.

\end{itemize}

\begin{enumerate}
\item When did you go? (i.e. month/season)

\item How did you get there? (mode of transport)

\item Why did you go? Did you do anything else there?

\item Who did you go with? (age/sex/relationship)

\item Who did you stay with? (age/sex/relationship)

\item Do you have any children there? (number/age)

\item Do you have a lover there? (only if appropriate)

\end{enumerate}

%----------------------------------------------------------------------------------------

\subsection{Lifetime range}

\begin{itemize}

\item Create a list of 50 locations in the broader region.

\item For each location, ask the participant ``Have you ever visited X''.  If they respond affirmatively, ask ``Have you visited X only once? a few times? or many times?''  Indicate 0 for ``never'', 1 for ``once'', 2 for ``a few times'', 3 for ``many times'', and 4 if they state that the location was a previous residence.

\end{itemize}


%----------------------------------------------------------------------------------------

\section{Navigation}

\subsection{Pointing task} % 

\begin{itemize}

\item Select 10 well-known locations spread throughout the local region.  These locations should be at least 10km away, and make sure that the target and distance are balanced such that there is only one correct answer.

For each location ask participants:

	\begin{itemize}

	\item When they most recently visited that location

	\item to indicate the bearing to each location
	
	\end{itemize}

\item Accurately measuring the exact bearing of participants' points is a difficult task.  There are two preferred options:

	\begin{enumerate}

	\item Using a tripod mounted Brunton compass, teach the participant that the arrow should point exactly where they would point with their finger.  The participant will then rotate the compass to the correct location.  Regularly challenge participants to ensure that the indicated bearings are consistent with bearings indicated by simple pointing.

	\item Using a handheld compass (either an Autohelm fluxgate or Suunto) ask the participant to naturally point to the location.  Stand directly in front of the participant at least 10' away to ensure you are not blocking their view of any potential cues.  Lineup the compass along the participant's finger and dominant eye (need to determine eye dominance before beginning task).  Since you are facing the participant, you need to either reverse the compass or the reading to record the indicated bearing.  
	
	\end{enumerate}


\end{itemize}

%-----------------------------------------

\subsection{Imaginary pointing task}

\begin{itemize}

\item Select 5 pairs of well-known locations spread throughout the local region.  These locations should be at least 10km away, and make sure that the target and distance are balanced such that there is only one correct answer.

For each pair of locations ask participants:

\item When they most recently visited each location

\item to indicate the bearing to location B while imagining that they are standing in location A.

\item Use one of the two methods noted in the discussion of the standard pointing task for this task as well.

\end{itemize}

%-----------------------------------------

\subsection{Out of camp pointing task}

\begin{itemize}

\item During focal follows, ask participants to point to home at random times.

\item Use the second method discussed in the standard pointing task section above to record indicated bearings.

\item Note the indicated bearing, GPS position, and who is in the group.

\end{itemize}

%-----------------------------------------

\subsection{Cue preference: Map Drawing}

\begin{itemize}

\item ``Draw a map of X village (ideally a very well-known village that is not the current location)''

\item Ideally, record the participant drawing the map with a handheld camera.

\item If a video camera is not available: 
	\begin{itemize}

	\item describe the order in which participants draw the map. (i.e. which features are drawn first?)
	
	\item note all distal and proximal landmarks drawn
	
	\item code accuracy of spatial relations and complexity (how?)	
	\end{itemize}

 
\end{itemize}

%-----------------------------------------

\subsection{Cue preference: Directions}

\begin{itemize}

\item Select three pairs of locations.  These could be villages, major landmarks, market towns, or any other well-known places in the local region.

\item For each pair of locations, ask participants: ``Pretend I am in location A and would like to walk to location B. How would I get there?''

\item Ideally, record the participant giving directions with a handheld camera.  We are interested in both the cues used in their verbal description as well as any supporting body-language (e.g. pointing towards the goal)

\item If a video camera is not available: 
	\begin{itemize}

	\item attempt to transcribe the participants' directions (including notes of non-verbal communication)
	
	\item note all distal and proximal landmarks used by the participants
	
	\item code accuracy of spatial relations and complexity (how?)	
	\end{itemize}
	
\item Alternatively.  You could ask the participant to draw the directions from A to B with paper and pencil.

\end{itemize}

%----------------------------------------------------------------------------------------

\section{Additional Questionnaires} % 

\subsection{Spatial Anxiety} % You don't need to make a \newexperiment if you only plan on referencing it once

\emph{Wayfinding anxiety:}

\begin{enumerate}

\item When you are someplace you don't know well, are you concerned about getting lost?  Or are you not concerned about getting lost, even in a new place?

\item When going to a place you don't know, would you feel safer going with others or would you feel as comfortable going by yourself?

\item If you made a wrong turn when you were out alone and didn't recognize where you were, would you be concerned that you might not find your way home?  Or would you be sure you would find?

\item Would you feel comfortable trying a new route that you thought would be shorter, even if no one had traveled it before? Or would you feel anxious to try it?

\item Should parents worry about their children becoming lost when they travel without supervision? Or is there nothing to worry about?

\end{enumerate}

\emph{Travel anxiety:}

\begin{enumerate}

\item Do you worry an animal could attack you while out in the bush? Or is that not a danger?

\item While traveling alone, if you see a stranger from a different tribe do you hide from them? Or do you go to greet the stranger?

\item If you become injured while alone in the bush there is nobody to help you. Does this worry you when you travel alone? Or does it not concern you?

\item  Is it safer to stay company when you need to sleep in the bush, or is it no different than sleeping alone?

\item If you see the weather becoming worse after you begin a trip do you return home for shelter or do you continue on your trip?

\end{enumerate}

%----------------------------------------------------------------------------------------
\section{Pilot}

\subsection{Perceptual biases under fear}

\begin{itemize}
\item \emph{height bias}:  How tall is that point on the tree?  

\begin{itemize}

\item Find a distinguishable branch on a tree, measure its height (with Brunton?) 

\item Ask person to imagine sitting on it

\item Ask person to estimate height: stop you as you walk backwards as far as tree is tall

\end{itemize}

\item \emph{distance bias}:  How far is it across the pool of death?

\begin{itemize}

\item mark out a circle, fill with brush and knives with blades up

\item Ask person to imagine jumping across it 

\item Ask person to estimate distance, as you walk backwards, as above

\end{itemize}

\item NOTES: The tree has a potential bias since men climb trees.  The
  pool of death has a potential bias because men can jump farther, use
  height as a covariate?  
  
\end{itemize}

%----------------------------------------------------------------------------------------

\end{document}